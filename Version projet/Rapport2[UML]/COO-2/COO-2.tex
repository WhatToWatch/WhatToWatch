\documentclass[10pt,a4paper]{article}
\usepackage[utf8]{inputenc}
\usepackage{amsmath}
\usepackage{amsfonts}
\usepackage{amssymb}
\begin{document}
\begin{center}
\Huge{\textbf{Choix 2}}

\large{\textbf{Olivier Bronchain ...}}

\end{center}

\section{Introduction}
	Durant ces dernières semaines, il nous a été demandé de modeliser les classes qui se retrouveraient dans le code de notre application WhatToWatch. Pour se faire nous avons d'abord écrit des Users Stories, ensuite nous avons ecrit des cartes CRC. Avec celles-ci nous avons pu modeliser nos schémas URM relationelle et puis sequencielle.
\section{Creation des diagrammes}
\subsection{Users-Stories}
	La premiere étape pour realiser nos classes était de savoir exactement quelles fonctionalités notre application devait remplir. En se basant sur nos faits élémentaires, sur nos envies et les consignes fournies, nous avons écrit ces histoires d'ulisateurs. Celles-ci sont disponibles dans un autre fichier. Pour permettre une meilleur visibilité pour la suite des opérations, il a été choisit de les divisées en plusieurs catégories:
	\begin{itemize}
		\item L'interaction avec l'utilisateur
		\item Les recherches faites par l'utilisateur
		\item L'interaction entre l'utlisateur et les films
		\item La gestion des utilisateurs
		\item La gestion des films
		\item Les droits relatifs aux administateurs
	\end{itemize}
\subsection{Schema relationel UML}

\subsection{Schema sequentielle UML}
\end{document}